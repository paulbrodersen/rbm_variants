% Created 2018-10-15 Mon 14:28
\documentclass[11pt]{article}
\usepackage[utf8]{inputenc}
\usepackage[T1]{fontenc}
\usepackage{fixltx2e}
\usepackage{graphicx}
\usepackage{longtable}
\usepackage{float}
\usepackage{wrapfig}
\usepackage{rotating}
\usepackage[normalem]{ulem}
\usepackage{amsmath}
\usepackage{textcomp}
\usepackage{marvosym}
\usepackage{wasysym}
\usepackage{amssymb}
\usepackage{xcolor}
\usepackage{hyperref}
\hypersetup{
    colorlinks,
    linkcolor={red!50!black},
    citecolor={blue!50!black},
    urlcolor={blue!80!black}
}
\usepackage[round]{natbib}
\tolerance=1000
\author{Paul J N Brodersen, Colin J Akerman}
\date{\today}
\title{Restricted Boltzmann machines with asymmetric forward and backward weights learn to reconstruct inputs}
\hypersetup{
  pdfkeywords={},
  pdfsubject={},
  pdfcreator={Emacs 25.1.1 (Org mode 8.2.10)}}
\begin{document}

\maketitle

\begin{abstract}
In canonical Restricted Boltzmann machines (RBM), connections between
units are bi-directional, i.e. the connection from unit $i$ to unit
$j$ is identical to (and has the same weight as) the connection from
unit $j$ to unit $i$. Here, we relax this symmetry constraint and
explore RBM variants where the forward and backward connection weights
are initialised and updated independently from one another. We show
that RBMs with asymmetric connectivity can still successfully be
trained with contrastive divergence. Surprisingly, this holds true
even if the forward, visible-to-hidden layer weights are kept constant
and only the backward, hidden-to-visible layer weights are updated
during training.
\end{abstract}

A restricted Boltzmann machine (RBM) is an artificial neural network
that in its simplest form consists of two layers of Boltzmann units.
A Boltzmann unit is a stochastic binary unit that is active with a
probability that is a function of the sum its connection weights to
other active units and a bias term. The connectivity in the network
is limited (``restricted'') to connections between units from different
layers. The activity of a Boltzmann units is hence independent of the
activity of other units in the same layer, and only depends on the
activities of units in the other layer.

RBMs were originally developed by Paul Smolensky as a way to learn a
generative graphical model that abstracts higher order features from a
set of training inputs \citep{Smolensky1986a}. In this model, the input
features are represented by the units in the so-called ``visible''
layer. The higher order, latent features are represented by units in
the ``hidden'' layer. The presence or absence of a feature is
represented by the activity of the corresponding Boltzmann unit, which
is either ``ON'' or ``OFF''. The central idea of the RBM algorithm is
to learn a set of weights for the connections between the two layers
such that the hidden unit activity patterns are consistent with
visible unit activity patterns for any given input. This consistency
is achieved if for a set of inputs, the visible unit activities result
in hidden unit activities that on the next iteration reconstruct the
visible unit activities again. To reflect this consistency criterion
the RBM was originally referred to as a ``Harmonium''. As a
consequence, RBMs are always also auto-encoders.

Subsequently, Geoff Hinton and colleagues discovered a fast variant of
the original learning algorithm (contrastive divergence
\citep{Hinton2002,Carreira-Perpinan2005}) such that this method could
be applied to complex, non-trivial inputs. They showed that the hidden
units would learn to encode higher order features (e.g. the strokes in
images of handwritten digits in the MNIST data set) and they
demonstrated that the activity patterns in small hidden layers could
form well-separated, compressed representations of the corresponding
input patterns \citep{Hinton2006b}. As a consequence, stacks of
pre-trained RBMs form good initialisations for deep belief nets. Deep
belief nets are artificial neural networks with multiple (\textgreater
3) layers of hidden units that are trained in classification of
regression tasks using using backpropation of errors (``backprop'')
\citep{Rumelhart1986}. Transporting error gradients in these networks
is initially often difficult, thus significantly slowing (or even
halting) gradient descent. Hinton and colleagues showed that
pre-training with the RBM algorithm significantly improved performance
of these networks \citep{Hinton2006a,Hinton2006b}. Interestingly, the
subsequent fine tuning of the hidden layer connectivity with backprop
turned out to be comparatively small \citep{Erhan2010}.

Although artificial neural networks trained with the RBM algorithm or
the backprop algorithm were originally envisioned as models for
biological neuronal networks, both algorithms postulate a symmetric
connectivity between neurons. In an RBM, the connection from unit $i$
to unit $j$ is identical to (and has the same weight as) the
connection from unit $j$ to unit $i$. In backprop, errors w.r.t the
activity of a neuron $i$ are backpropagated to the neuron $j$ in the
layer below proportional to the weight of the (forward) connection
from $j$ to $i$. %
This symmetry in the connectivity is in contrast to biological
neuronal networks, where the vast majority of synapses transmit
information in one direction (with the exception of electrical gap
junctions), and only an exceedingly small percentage of neurons appear
bi-directionally connected to each other (although interestingly, in
the cortex this proportion seems to be higher than chance, at least
for nearby neuron pairs). %
Recently, Lillicrap et al. have shown that for backprop this symmetry
constraint can be abolished \citep{Lillicrap2014}. In their backprop
variant called ``feedback alignment'', successive layers are connected
by distinct forward and backward weights. The forward weights are
learnt by backpropagating errors arising at the output layer via the
(distinct!) backward weights to the lower layers. The randomly
initialised backward weights are kept fixed for the entirety of
training. Despite the random feedback error, such a network rapidly
learns to performs the task as well as a network trained with the
canonical backpropation algorithm.

In this paper, we investigate whether the symmetry constraint on the
connectivity of an RBM can similarly be relaxed while retaining a high
performance, and we show that this is indeed the case in two different
ways: %
First, we show that an RBM with distinctly initialised forward and
backward weights can still successfully learn the task. However, as
the same weight updates are applied to the forward and backward
weights, one might argue that this is unsurprising as forward and
backward weights can become sufficiently similar with learning as long
as the cumulative weight changes are sufficiently large. We hence
demonstrate that this finding remains true in a sparse network that
does not have any bi-directional connections. %
Second, we show that it is sufficient to update the backward weights
in order to learn to reconstruct a set of inputs. This is reminiscent
of feedback alignment, only that in this case the forward weights are
kept constant and the backward weights are learnt. %
These findings have significant implications for the development of
novel RBM variants and for the plausibility of RBM-like networks as
models for biological neuronal networks.

\section{Methods}

\subsection{Data acquisition and pre-processing}

All experiments were carried out using images from the MNIST database
of handwritten digits, which was retrieved from
\url{http://yann.lecun.com/exdb/mnist/}. The original train and test
split of the data set was retained. Example inputs and example
reconstructions correspond to the first 100 samples in the test data
set. As the states of the visible units are constrained to be between
0 and 1, all inputs were normalized to this range by dividing image
pixel intensities by 255.

\subsection{The restricted Boltzmann machine}

All RBM networks consisted of two layers, a visible layer with 784
units corresponding to the 784 pixels in a single MNIST image, and a
hidden layer of 400 units, i.e. a layer with about half as many units
as in the visible layer. A hidden layer with fewer units (and hence
features) than present in the visible layer ensures that the
transformation learnt by the RBM does not simply correspond to the
identity transform, and demonstrates the ability of the RBM to
compress the set of input features to a set of hidden features.

All hidden layer units were modelled as Boltzmann units. A Boltzmann
unit is in the active state with the probability given by the logistic
function of the sum of its inputs from visible units and a bias term:
\begin{equation}
  p_j = p(h_j = 1) = \sigma(b_j + \sum_i v_i * w_{ij})
\end{equation}

Here, $i$ and $j$ index the visible units $v_i$ and hidden units $h_j$,
respectively, $b$ is the bias term associated with each unit, and
$w_ij$ is the weight of the connection from $v_i$ to $h_j$. The function
$\sigma$ represents the logistic function, which is defined as
\begin{equation}
  \sigma(x) = \frac{1}{1 + e^{-x}}.
\end{equation}

Visible layer units were modelled similarly:
\begin{equation}
  p_i = p(v_i = 1) = \sigma(b_i + \sum_j h_j * w_{ji}).
\end{equation}

However, unlike hidden layer units which transmitted their binary
state to visible units, visible units transmitted the probability of
being active $p_i$ instead of their state $v_i$. Such a procedure has
previously been suggested to reduce sampling noise and thus speed up
learning \citep{Hinton2012}. Strictly speaking, the visible units were
hence modelled as logistic units, and not as Boltzmann units.

\subsection{Weight and bias updates}

Weights and biases were updated according to the canonical RBM update rules:
\begin{align}
  \label{eq:rbm_update_equations}
  \Delta w_{ij} = \Delta w_{ji} &\propto \langle p_{i}h_{j} \rangle_{\text{Data}} - \langle p_{i}h_{j} \rangle_{\text{Model}} \\
  \Delta b_{i}                  &\propto \langle p_{i}      \rangle_{\text{Data}} - \langle p_{i}      \rangle_{\text{Model}} \\
  \Delta b_{j}                  &\propto \langle h_{j}      \rangle_{\text{Data}} - \langle h_{j}      \rangle_{\text{Model}}
\end{align}

The variable $\eta$ correponds to the learning rate. As before, $p_i$
is the probability of the visible unit $i$ to be active, $h_j$ is the
activity state of hidden unit $j$, and $w_{ij}$ is the weight of the
connection from $i$ to $j$. The term $\langle p_{i}h_{j} \rangle$ with
the subscript ``Data'' thus corresponds to the co-activity state of
visible unit $i$ and hidden unit $j$ when the visible units are
clamped at the input sample values. The subscript ``Model'' indicates
the co-activity state when the latter is not the case.

The ``Data'' and ``Model'' activity states were sampled using contrastive
divergence. Specifically, the RBM was initialised with the visible
unit states set to the values of an input sample, and on that basis
the hidden unit states were computed. These unit states formed the
basis of the ``Data'' samples. To sample the ``Model'' states, the network
was let to evolve for a number of $CD$ additional backward and forward
passes of the activity:
\begin{align}
  \label{eq:cd_update_equations}
  \Delta w_{ij} = \Delta w_{ji} &= \eta \Big[\langle p_{i}h_{j} \rangle_{0} - \langle p_{i}h_{j} \rangle_{CD}\Big] \\
  \Delta b_{i}                  &= \eta \Big[\langle p_{i}      \rangle_{0} - \langle p_{i}      \rangle_{CD}\Big] \\
  \Delta b_{j}                  &= \eta \Big[\langle h_{j}      \rangle_{0} - \langle h_{j}      \rangle_{CD}\Big]
\end{align}
All experiments were performed with $CD=3$ contrastive divergence
iterations, and a learning rate $\eta = 0.01$.

% When training RBM networks with constant forward weights, backward
% weights or both, we furthermore enforced a target probability $q_j$ in
% the hidden units activity by adding a small penalty $\delta$ to the
% weight and hidden unit bias updates:

% \begin{align}
%   \label{eq:sparsity_penalty}
%   \delta_{j}                    &= \theta \Big[q_{j} - \langle h_{j} \rangle\Big] \\
%   \Delta w_{ij} = \Delta w_{ji} &= \eta \Big[\langle p_{i}h_{j} \rangle_{0} - \langle p_{i}h_{j} \rangle_{CD} + \delta_{j}\Big]\\
%   \Delta b_{j}                  &= \eta \Big[\langle h_{j}      \rangle_{0} - \langle h_{j}      \rangle_{CD} + \delta_{j}\Big]
% \end{align}

% This ensured that the majority of hidden units were not always active
% or inactive, which otherwise tended to be the case in these networks.
% The sparsity cost $\theta$ was set to unity.

\subsection{Weight and bias initialisation}

Unless stated otherwise, biases were initialised with 0. All weights
were initialised with values drawn from the normal distribution
$N(0.0, 0.1)$.

% Fixed weights were drawn from the normal distribution $N(0.0,
% 1.0)$. Fixed weights were (initially) thus much larger than learnt
% weights, ensuring that even after learning, the activities of units
% receiving inputs only via fixed weights were still sensitive to the
% input from units in the other layer, and not completely dominated by
% their biases.

\subsection{Software implementation}

The code to reproduce all experiments shown in the figures below can
be accessed at\url{https://github.com/paulbrodersen/rbm_variants}.
The MNIST data was loaded into memory using the python-mnist
package. All RBM variants were implemented in python. Numerical
computations relied in part on functions from the numpy package. All
visualisation were made using the matplotlib package.

\section{Results}

\subsection{Restricted Boltzmann machines with independent and asymmetric forward and backward weights learn to reconstruct inputs}

A canonical RBM forms an undirected graph: if neuron $i$ is connected
to neuron $j$ with weight $w_{ij}$ then neuron $j$ is connected to
neuron $i$ with weight $w_{ji} = w_{ij}$. We are interested in the
question if network architectures that are less constrained in their
connectivity are still able to learn and perform the same
computation. As a first pass, we hence evaluated the performance of
RBMs with independently initialised forward and backward weights
(i.e. $w_ij \neq w_{ji}$) on a standard machine learning data set, the
MNIST database of handwritten digits. In analogy to directed graphs,
we call such an RBM a ``directed'' RBM. Fig.~\ref{fig:directed_rbm}
compares the performance of a canonical, undirected RBM network with
the performance of such a directed RBM with distinct forward and
backward weights. Clearly, the directed RBM is still able to learn to
reconstruct the MNIST digits, albeit with a marginally lower
performance.

\begin{figure}[H]
  \label{fig:directed_rbm}
  \centering
  \includegraphics[width=\linewidth]{../figures/figure_1/f1.pdf}
  \caption{\footnotesize
    \textbf{Distinct, asymmetric forward and backward weights are sufficient for learning:}
    A canonical RBM and a directed RBM were trained to reconstruct
    images in the MNIST handwritten digits database.
    Each RBM consisted of 784 visible units and 400 hidden units.
    Biases and connection weights were updated using contrastive
    divergence with 3 iterations and a learning rate $\eta=0.01$.
    a) Images of the first 100 characters in the MNIST test set.
    b) Cartoon representation of a canonical RBM. The backward weight
    matrix $w^T$ is the transpose of the forward weight matrix $w$.
    c) Cartoon representation of a directed RBM. The forward weight
    matrix $w_F$ is distinct of the backward weight matrix $w_B$.
    d) Canonical RBM example reconstructions of the first 100
    characters in the test set after 10 epochs of learning on the
    training set.
    e) The corresponding directed RBM example reconstructions.
    f) Mean squared error between the input images and RBM
    reconstructions for the test set plotted against the number of
    training samples.
    g) Final mean squared error after 10 epochs. Error bars correspond
    to the standard deviation around the mean ($n = 10$).
  }
\end{figure}

However, as we have not changed the weight update rules for the
directed RBMs, the same weight updates are applied to the forward and
backward weights. As sufficiently large weight changes accumulate over
time, the forward and backward weight matrices become more similar to
one another. This effect can be seen in
Fig.~\ref{fig:weight_alignment}, which plots forward weights against
backward weights before and after training. Clearly, the forward and
backward weights align to one another with learning, as indicated by
the elongation of the data along the $x=y$ axis.

\begin{figure}[H]
  \label{fig:weight_alignment}
  \centering
  \includegraphics[width=1.0\linewidth]{../figures/supplementary_1/s1.pdf}
  \caption{\textbf{Forward and backward weights in directed RBMs
      become more similar with training}
    Forward weights $w_{ij}$ plotted against the corresponding
    backward weights $W_{ji}$ before and after training a directed RBM
    network for 10 epochs on MNIST.
  }
\end{figure}

Although the weight alignment tends to be small for sensibly
initialised network architectures, it could be the driver of learning
in the directed RBMs by effectively reducing the directed networks to
approximately undirected networks. To show that this is not the case,
we compared three sparsely connected RBM variants, namely (1) a
canonical RBM (i.e. with purely bi-directional connections), (2) a
directed RBM with independently initialised forward and backward
connections containing a mix of bi-directional and uni-directional
connections, and (3) a directed RBM constrained to have purely
uni-directional connections.
% As can be seen in Fig.~\ref{fig:unidirectional_connections}, the
% performance of a directed RBM constrained to have purely
% uni-directional connections is comparable to the performance of a
% directed RBM containing a mix of bi-directional and uni-directional
% connections. As before, both directed RBM variants performed slightly
% worse than the canonical RBM.
As can be seen in Fig.~\ref{fig:unidirectional_connections}, RBM
performance decreases with a decrease in the number of bi-directional
connections. However, the absolute performance decrease from an RBM
with purely bi-directional connections to a RBM with purely
uni-directional connections is with 1.3\% overall very small.
These findings show that weight alignment is not necessary
for learning but that learning uni-directional connections is
sufficient for training an RBM. Perfect symmetry in the forward and
backward connectivity thus may confer some advantage in learning, but
is clearly not necessary.

\begin{figure}[H]
  \label{fig:unidirectional_connections}
  \centering
  \includegraphics[width=0.8\linewidth]{../figures/figure_2/f2.pdf}
  \caption{\footnotesize
    \textbf{Uni-directional connections are sufficient for learning:}
    Three sparsely connected RBM variants with differing proportion of bi-directional connections were trained for 10 epochs on the MNIST handwritten digit data set.
    As before, each RBM consisted of 784 visible units and 400 hidden units,
    and biases and connection weights were updated using contrastive divergence with 3 iterations and a learning rate $\eta=0.01$.
    However, the connectivity between units in different layers was sparse ($p=0.5$),
    and the three variants differed in the proportion of
    bi-directional connections ($w_{ij} \neq 0 \land w_{ji} \neq 0$),
    which was set to 1.0 (first column), 0.25 (second column), and 0.0
    (third column).
    a)~-~c) Cartoon representation of the sparse RBM variants tested.
    % : a canonical RBM, a directed RBM with randomly initialised forward
    % and backward connections, and a directed RBM with purely
    % uni-directional connections.
    d)~-~f) Corresponding example reconstructions of the first 100 characters in the test set after 10 epochs of learning on the training set.
    g)~-~i) Corresponding size of the forward weights $w_{ij}$ versus the corresponding backward weights $w_{ji}$ for an example network after training for 10 epochs.
    j) Mean squared error between the input images and RBM reconstructions for the test set plotted against the number of training samples.
    k) Final mean squared error after 10 epochs. Error bars correspond to the standard deviation around the mean ($n = 10$).
  }
\end{figure}

\subsection{Updating the backward weights is sufficient for learning}

In the previous section, we showed that learning uni-directional
connections is sufficient to train an RBM. Furthermore, as alluded to
in the introduction, Lillicrap et al. previously showed that neural
networks can be trained with backpropagation of errors via fixed
backward weights. By analogy, it thus seemed possible that RBMs could
be trained by updating only one set of connections, i.e. either the
forward or the backward weights. To test this hypothesis, we compared
five RBM variants: (1) the canonical RBM network, (2) a directed RBM,
where as before both, the forward and the backward weights, were
updated, 3) a directed RBM, where the forward weights were kept fixed
and the backward weights were updated, 4) a directed RBM, where the
forward weights were updated and the backward weights were kept fixed,
and (5) a directed RBM, where both, forward and backward weights, were
kept fixed (and only the unit biases were updated). As shown in
Fig. 4, updating only the backward weights is sufficient for learning
to reconstruct the inputs with a performance that is comparable to
(albeit smaller than) the performance of a canonical
RBM. Interestingly, the ``inverse'' variant, i.e. learning the forward
weights while keeping the backward weights fixed also performs better
than just learning the biases; however, the performance of this RBM
variant is substantially lower than in the canonical case.

\begin{figure}[H]
  \label{fig:backward_rbm}
  \centering
  \includegraphics[width=\linewidth]{../figures/figure_3/f3.pdf}
  \caption{\footnotesize
    \textbf{Updating the backward weights is sufficient for learning:}
    A canonical RBM and four directed RBM variants were trained for 10 epochs on the MNIST handwritten digit data set.
    As before, each RBM consisted of 784 visible units and 400 hidden
    units, and biases and connection weights were updated using contrastive
    divergence with 3 iterations and a learning rate $\eta=0.01$.
    a)~-~e) Cartoon representation of RBM variants tested:
    (a) a canonical RBM,
    (b) a directed RBM, where forward weights, backward weights, and unit biases were updated,
    (c) a directed RBM, where the forward weights were kept fixed,
    (d) a directed RBM, where the backward weights were kept fixed, and
    (e) a directed RBM, where both sets of weights were kept fixed and only the unit biases were updated.
    f)~-~j) Corresponding example reconstructions of the first 100 characters in the test set after 10 epochs of learning on the training set.
    k) Mean squared error between the input images and RBM reconstructions for the test set plotted against the number of training samples.
    l) Final mean squared error after 10 epochs. Error bars correspond to the standard deviation around the mean ($n = 10$).
  }
\end{figure}

\section{Discussion}

In summary, the symmetry constraint imposed on the forward and
backward weights in the canonical RBM is not necessary for
learning. Instead, even directed networks with strictly
uni-directional connections are still able to learn, as are networks,
in which only the backward weights are updated. %
These findings indicate that there is a family of network
architectures with diverse connectivity schemes that support RBM-like
learning, and this diversity, in turn, opens up the possibility that
some biological neuronal networks may learn to extract features from
their sensory inputs using RBM-like mechanisms. %
Nevertheless, it is worth noting that the canonical RBM outperforms
all other variants explored here. This indicates that bi-directional,
or even symmetric connections may bequeath a fundamental advantage for
learning certain relationships. Interestingly -- and as mentioned
above -- bi-directional connections -- while rare -- still seem to be
more common in the brain than expected by chance, and the findings
reported here might point towards reasons why that is the case.

There are several obvious avenues of future work. %
First, although the performance gain due to perfect symmetry is small,
it would stil be interesting to understand why it exists at all. %
Secondly, it remains to be seen if the RBM variants explored here
generalise to architectures with several hidden layers. Stacks of RBMs
are often used to initialise deep belief nets that are then trained
with backpropagation. Such an initialisation is known to often lead to
improved results, in particular if the amount of labelled training
data is limited. It would be interesting to see if these RBM variants
explored here can also form good initialisations for deep
architectures. %
Finally, the fact that updating the backward weights is sufficient for
learning frees up the forward weights to be used to learn other,
potentially unrelated functions of the input. As learning in feedback
alignment only requires updating the forward weights, it should be
possible to combine both learning rules. The resulting network would
learn to reconstruct and classify inputs simultaneously.

\pagebreak

\section{References}
\bibliography{references}
\bibliographystyle{apalike}
% Emacs 25.1.1 (Org mode 8.2.10)
\end{document}
